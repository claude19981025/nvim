\documentclass[12pt,a4paper]{article}
\title{A Complete Guide of Neovim}
\author{Claude Lu}

\usepackage{xcolor}     
\usepackage{float}
\usepackage{graphicx}
\usepackage{enumerate}
\usepackage{enumitem} %允許自定義bullet
\usepackage{amsmath, amssymb, amsfonts} %basic package for math expressions.
\usepackage{listings}
\usepackage[top=1in, bottom=1in, left=1in, right=1in]{geometry}
%表格必用
\usepackage{multirow}
\usepackage{hhline, booktabs}
%enable hyperlink
\usepackage{hyperref}
\usepackage{array}

%enable lstlisting
\usepackage{listings}

\usepackage{framed}

%自定義顏色
\definecolor{rublue}{HTML}{0036A7}
\definecolor{rured}{HTML}{D62718}
\definecolor{vscodegreen}{HTML}{4EC9B0}
\definecolor{vscodeblack}{HTML}{1E1E1E}
%\definecolor{vscodeblack}{HTML}{4D4D4D}
\definecolor{vscodeyellow}{HTML}{DCDCAA}
\definecolor{vscodeblue}{HTML}{569CD6}
%自定義命令
\newcommand{\bluetexttt}[1]{\textcolor{blue}{\texttt{#1}}}
\newcommand{\redtexttt}[1]{\textcolor{red}{\texttt{#1}}}
\newcommand{\redtext}[1]{\textcolor{red}{#1}}

\newcommand{\rt}[1]{\textcolor{red}{#1}}
\newcommand{\bt}[1]{\textcolor{blue}{#1}}
\newcommand{\vsgt}[1]{\textcolor{vscodegreen}{#1}}
\newcommand{\vsft}[1]{\textcolor{vscodeyellow}{#1}}
\newcommand{\vsbt}[1]{\textcolor{vscodeblue}{#1}}

\newcommand{\struct}[1]{\colorbox{vscodeblack}{\vsgt{#1}}}
\newcommand{\function}[1]{\colorbox{vscodeblack}{\vsft{#1}}}
\newcommand{\macro}[1]{\colorbox{vscodeblack}{\vsbt{#1}}}


\newcommand{\bs}[1]{\boldsymbol{#1}}


\lstset{
	basicstyle=\ttfamily, %use typewriter font
	keywordstyle=\color{blue},%關鍵字顏色
	commentstyle=\color{teal}, %註釋顏色 
	stringstyle=\color{red}, % Strings in red
	tabsize=4, % Set tab size to 4 spaces
	showspaces=false, % Do not show spaces
	showstringspaces=true, % Do not show string spaces
	breaklines=true, %自動換行
	frame=single %在代碼外加上外框
}
\begin{document}
\maketitle

\section{lua}
\begin{enumerate}
	\item Run lua by \texttt{:lua print("Hello")}
	\item put \bt{init.lua} under \textbf{\textasciitilde/.config/nvim}
	\item put \bt{.lua} file under \bt{\textasciitilde/.config/nvim/lua}
	\item One can import table (module) by \bt{require(``module")} (no need to add .lua)
	\item Better use \bt{require} with \bt{pcall} to increase robustness
	\item Use \bt{vim.cmd(``command'')} in lua to pass command to nvim
	\item Nvim supports \texttt{init.lua} as the configuration file, which should be put in \texttt{~/.config/nvim/}
	\item Lua modules should be put at \texttt{~/.config/nvim/lua}
\end{enumerate}
How to use lua
\begin{footnotesize}
\begin{verbatim}
-- run lua code
:lua print("Hello")

-- run an external lua file
:source ~/path/to/file.lua

-- Use Lua file on startup

\end{verbatim}
\end{footnotesize}

\subsection{Import Lua Modules}
\begin{enumerate}
	\item Modules are searched for under \bt{runtimepath}, any `\bt{.}' appeared is treated as \textbf{directory seperator}
	\item Neovim will search for the first module found
\end{enumerate}

\section{Variables}
\begin{enumerate}
	\item vim.g
	\item vim.b
	\item vim.w
	\item vim.t
	\item vim.v
	\item vim.env
\end{enumerate}

\section{runtimepath}
\begin{enumerate}
	\item 
	\item \texttt{:echo \&runtimepath} show runtimepath
\end{enumerate}

\begin{footnotesize}
\begin{verbatim}
-- in init.lua
-- vim.o is a global table
vim.o.runtimepath = vim.o.runtimepath..',/path/to/new/runtime/directory'
\end{verbatim}
\end{footnotesize}


\section{colorscheme}
\begin{footnotesize}
\begin{verbatim}
:colo -- show current active color scheme
:colo {name} -- load color scheme {name}
\end{verbatim}
\end{footnotesize}

\section{lazy.nvim}
\begin{footnotesize}
\begin{verbatim}
:checkhealth lazy
:echo stdpath('config')

-- add following code in our init.lua
require("config.lazy")
\end{verbatim}
\end{footnotesize}
\end{document}

